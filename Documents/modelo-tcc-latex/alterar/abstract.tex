% ====================================================================
% Abstract 
% ====================================================================
\noindent
SANTOS, Guilherme. \textbf{Análise e desenvolvimento de uma ferramenta para automação de captação de anúncios.}
Florianópolis, 2014. \pageref{nropaginas}f. Trabalho de Conclusão de Curso Superior de Tecnologia em
Análise e Desenvolvimento de Sistemas. Faculdade de Tecnologia do SENAI, Florianópolis, 2014.

\vspace{1cm}
\begin{resumo}[\textbf{ABSTRACT}]
 \begin{otherlanguage*}{english}
   The ad capture process doesn’t have a scientific bibliography available both the area itself, as for the others. Combining this fact with the large number of opportunities offered by the mobile application development market, this work aims to analyze and model the business process carried in the catchment vehicle ad print media; improve the process analyzed using BPMN; and the development of an Android application prototype that includes part of the model proposed process. To achieve the final results of the work, the use of research was necessary exploratory, to enable greater familiarity with the subject; Bibliographic to deepen theoretical knowledge regarding the main technologies of development platforms mobile; and qualitative, to gather information about the process that involves capturing of an ad. The results achieved with the work were: the diagrams of macroprocess BPMN and the current process and the proposed process model; case diagram uses and its specification of use cases addressed in the prototype; and the prototype of an Android application and web contemplating the use cases and part of the optimized process.

   \vspace{\onelineskip}
 
   \noindent 
   \textbf{Key-words}: \textit{BPM. Business Process. Mobile applications market. Android.}
 \end{otherlanguage*}
\end{resumo}


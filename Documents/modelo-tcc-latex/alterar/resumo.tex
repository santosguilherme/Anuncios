% ====================================================================
% Resumo 
% ====================================================================

\noindent
SANTOS, Guilherme. \textbf{Análise e desenvolvimento de uma ferramenta para automação de captação de anúncios.}
Florianópolis, 2014. \pageref{nropaginas}f. Trabalho de Conclusão de Curso Superior de Tecnologia em
Análise e Desenvolvimento de Sistemas. Faculdade de Tecnologia do SENAI, Florianópolis, 2014.

\vspace{1cm}
\setlength{\absparsep}{18pt} % ajusta o espaçamento dos parágrafos do resumo
\begin{resumo}
O processo de captação de anúncios não possui uma documentação bibliográfica acessível tanto para a própria área, como para as demais. Conciliando esse fato com o grande número de oportunidades oferecidas pelo mercado de desenvolvimento de aplicativos móveis, o presente trabalho tem como objetivo analisar e modelar o processo de negócio realizado na captação de anúncios de veículos de mídias impressas; aperfeiçoar o processo analisado utilizando BPMN; e o desenvolvimento de um protótipo de aplicativo Android que contemple parte do modelo do processo proposto. Para alcançar os resultados finais do trabalho, foi necessário o uso da pesquisa exploratória, a fim de possibilitar maior familiaridade com o tema; bibliográfica, para aprofundar o conhecimento teórico referente às principais tecnologias das plataformas de desenvolvimento móvel; e qualitativa, para o levantamento de informações sobre o processo que envolve a captação de um anúncio. Os resultados alcançados com o trabalho foram: os diagramas de macroprocesso e BPMN do processo atual e do modelo de processo proposto; diagrama de casos de usos e sua especificação dos casos de uso abordados no protótipo; e o protótipo de uma aplicação Android e \textit{web} que contempla os casos de uso e parte do processo otimizado.

 \textbf{Palavras-chave}: BPM. Processo de negócio. Mercado de aplicativos móveis. Android.
\end{resumo}
